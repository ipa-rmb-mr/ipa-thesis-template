\chapter{Comparison}
\label{chp:comparison}
In the following section three simulation tools will be presented. This work needs a simulation tool that offers sophisticated contact and grasping physics to generate meaningful data the neural network can be trained on. A lot of physics engines, such as game engines like Bullet, sacrifice precision for speed and stability when it comes to the calculation of contact forces.
%[Explain physics engines briefly and add references for further insight]

\section{Gazebo}
\label{sec:gazebo}
According to \cite{ivaldi_tools_2014}, Gazebo is the most used simulation tool within the robotics community. In version 7.0 it comes with four different physics engines, ODE, Bullet, DART and Simbody.

\section{V-REP}
\label{sec:vrep}
V-REP (V. 3.5.0) is a proprietary robot simulation software with integrated development environment \cite{noauthor_v-rep_nodate} that has been created by Coppelia Robotics. It utilizes four different physics engines, Bullet, ODE, Vortex Dynamics and Newton Dynamics. \cite{ivaldi_tools_2014} shows that it offers the most satisfying user experience.

\section{Klamp't}
\label{sec:klampt}
Klamp't is an open-source software package that has been developed at Indiana University since 2009. It is meant for simulation, modeling, planning and optimization of robot systems, especially in terms of manipulation and locomotion \cite{inaba_robust_2016}.
The current version 0.8.1 makes use of the ODE pyhsics engine.

\newpage

\section{Comparison}
\label{sec:comparison}
\begin{table}[H]
\centering
\begin{tabular}{|l|c|c|c|c|c|c|}
	\hline
			& ROS Interface & API & Friction & Collision & Documentation & Community\\
	\hline
	Gazebo	& & & & & &\\
	V-REP	& & & & & &\\
	Klamp't	& & & & & &\\
	\hline
\end{tabular}
\caption{Comparison}
\label{tab:comparison}
\end{table}

\section{Result}
\label{sec:result}
Table \ref{tab:comparison} clearly shows that V-REP is well suited for the current task.